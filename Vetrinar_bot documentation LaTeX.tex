\documentclass[a4paper,12pt]{article} % Document class and basic formatting
\usepackage[utf8]{inputenc}           % UTF-8 input encoding
\usepackage[T2A]{fontenc}             % Cyrillic font encoding
\usepackage[russian]{babel}           % Russian language support
\usepackage{hyperref}                 % For hyperlinks
\usepackage{graphicx}                 % For images
\usepackage{lmodern}                  % Improved font rendering
\usepackage{fancyvrb}                 % Fancy verbatim (code-like) environments
\usepackage{listings}                 % Code listings
\usepackage{xcolor}                   % Color support (for code highlighting)
\usepackage{caption}                 % Caption formatting
\usepackage{geometry}                % Page margin setup
\geometry{margin=2.5cm}              % Set 2.5cm margins


\title{Документация Telegram-бота для ГАУ ТО \\ «Тобольский ветцентр»} % Project title
\author{Разработано: Попова Виктория Викторовна} % Author name
\date{\today }% Date of document generation

\begin{document}

\maketitle
\tableofcontents
\newpage

\section{Введение} % Introduction section — explaining what the bot is and who it's for

Данный Telegram-бот создан для автоматизации взаимодействия с потенциальными сотрудниками и студентами, заинтересованными в прохождении практики или трудоустройстве в государственном автономном учреждении Тюменской области «Тобольский межрайонный центр ветеринарии».

\section{Общая архитектура} % Architecture section — explains the components of the project
\subsection{Основные компоненты}
\begin{itemize}
    \item \texttt{main.py} — основной скрипт, реализующий бизнес-логику Telegram-бота.  % main.py — the main bot logic, handles Telegram interaction and callbacks
    \item \texttt{database.py} — модуль для взаимодействия с базой данных SQLite. % database.py — handles database operations (SQLite)
    \item \texttt{check\_db.py} — вспомогательный скрипт для ручной проверки данных в базе. % check_db.py — optional helper script for manually inspecting stored data
\end{itemize}

\subsection{Иллюстрации в интерфейсе бота} % These figures demonstrate images that the bot sends to users during interaction
\begin{figure}[h!]
\centering
\includegraphics[width=0.6\textwidth]{guarantees.jpg}
\caption{Изображение, отправляемое ботом при выборе «Преимущества работы в нашем учреждении»} % Figure 1: Visual explanation of work advantages
\end{figure}

\begin{figure}[h!]
\centering
\includegraphics[width=0.6\textwidth]{thestory.jpg}
\caption{Изображение, отправляемое ботом при выборе «Истории успеха ветеринарных специалистов»} % Figure 2: Success stories of veterinary specialists
\end{figure}

\newpage
\subsection{База данных}% This code block shows the SQL structure of the 'applicants' table
Структура таблицы \texttt{applicants}, в которую сохраняются данные анкетирования: % where data collected from users (via questionnaire) is stored

\begin{lstlisting}[language=SQL, basicstyle=\ttfamily\small, backgroundcolor=\color{gray!10}]
CREATE TABLE applicants (
    id INTEGER PRIMARY KEY AUTOINCREMENT,
    telegram_id INTEGER,
    fio TEXT,
    contact TEXT,
    age TEXT,
    edu TEXT,
    year TEXT,
    exp TEXT,
    about TEXT
);
\end{lstlisting}

\section{Функционал бота}
\subsection{Команды}% Bot command list — describes what commands users (and admins) can issue
\begin{itemize}
    \item \texttt{/start} — приветствие и главное меню. % /start — starts the bot
    \item \texttt{/getdb} — отправка текущей базы данных администратору. % /getdb — sends DB file to admin
    \item \texttt{/showdb} — текстовый вывод всех записей из базы (ограничен по длине).% /showdb — shows data in chat (limited length)
\end{itemize}

\subsection{Интерактивное меню} % Description of the interactive menu options available after /start    
Доступные действия:
\begin{itemize}
    \item Чем мы занимаемся и какова наша миссия. % Users can learn about the center and it's mission
    \item Преимущества работы в нашем учреждении. % Users can see work benefits
    \item Истории успеха ветеринарных специалистов.% Users can read success stories
    \item Как присоединиться к нашей команде. % Users can fill questionnaire for employment
\end{itemize}

\section{Ключевые функции \texttt{main.py}} % Table of key functions in main.py
% Each row explains the purpose of the function: user interaction, data handling, or admin commands
\begin{table}[h!]
\centering
\renewcommand{\arraystretch}{1.3}
\begin{tabular}{|p{4,5cm}|p{10cm}|}
\hline
\texttt{start(message)} & Обработка команды \texttt{/start}, отображение главного меню. \\ \hline
\texttt{callback\_handler(call)} & Обработка нажатий на кнопки и маршрутизация действий. \\ \hline
\texttt{main\_handler(message)} & Диалог с пользователем в процессе заполнения анкеты. \\ \hline
\texttt{handle\_contact(message)} & Сохранение контактной информации пользователя. \\ \hline
\texttt{send\_db\_file(message)} & Отправка файла базы данных в ответ на \texttt{/getdb}. \\ \hline
\end{tabular}
\caption{Основные функции скрипта \texttt{main.py}}
\end{table}

\section{Установка и запуск} % Setup and installation instructions (for developers or HR staff) 
\begin{enumerate}
    \item Установите зависимости: % Step 1: Install required Python libraries
    \begin{lstlisting}[language=bash]
pip install pyTelegramBotAPI python-dotenv
    \end{lstlisting}

    \item Создайте файл \texttt{.env} со следующим содержимым: % Step 2: Create .env file with your personal Telegram Bot API key
    \begin{lstlisting}[language=bash]
API_KEY=your_token_here
    \end{lstlisting}

    \item Запустите бота: % Step 3: Run the bot with Python
    \begin{lstlisting}[language=bash]
python main.py
    \end{lstlisting}
\end{enumerate}

\section{Заключение} % The bot simplifies HR work and student onboarding
Бот существенно упрощает процесс первичного контакта с потенциальными соискателями и студентами. Он позволяет автоматизировать сбор информации, предоставить ключевые сведения об учреждении и упростить документооборот.

\textbf{Планы на развитие:} % Future improvements:
\begin{itemize}
    \item Интеграция с электронной почтой для уведомлений HR; % - Email notifications for staff
    \item Интеграция с CRM-системами; % - CRM integration
    \item Возможность прикрепления резюме соискателей вместо анкетирования. % - Resume upload support (instead of plain forms)
\end{itemize}

\end{document}
